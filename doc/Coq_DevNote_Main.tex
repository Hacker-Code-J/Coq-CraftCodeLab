\documentclass[12pt,openany]{book}

% Preamble with packages, custom commands, etc.
\usepackage[utf8]{inputenc} % Allow input to be UTF-8
\usepackage{commath, amsmath, amsthm}
\usepackage{polynomial}
\usepackage{enumerate}
\usepackage{soul} % highlight
\usepackage{lipsum}  % Just for generating text

% Theorem
\newtheorem{axiom}{Axiom}[chapter]
\newtheorem{theorem}{Theorem}[chapter]
\newtheorem{proposition}[theorem]{Proposition}
\newtheorem{corollary}{Corollary}[theorem]
\newtheorem{lemma}[theorem]{Lemma}

\theoremstyle{definition}
\newtheorem{definition}{Definition}[chapter]
\newtheorem{remark}{Remark}[chapter]
\newtheorem{exercise}{Exercise}[chapter]
\newtheorem{example}{Example}[chapter]
\newtheorem*{note}{Note}

% Colors
\usepackage[dvipsnames,table]{xcolor}
\definecolor{titleblue}{RGB}{0,53,128}
\definecolor{chaptergray}{RGB}{140,140,140}
\definecolor{sectiongray}{RGB}{180,180,180}

\definecolor{thmcolor}{RGB}{231, 76, 60}
\definecolor{defcolor}{RGB}{52, 152, 219}
\definecolor{lemcolor}{RGB}{155, 89, 182}
\definecolor{corcolor}{RGB}{46, 204, 113}
\definecolor{procolor}{RGB}{241, 196, 15}
\definecolor{execolor}{RGB}{90, 128, 127}

% Fonts
\usepackage[T1]{fontenc}
\usepackage[utf8]{inputenc}
\usepackage{newpxtext,newpxmath}
\usepackage{sectsty}
\allsectionsfont{\sffamily\color{titleblue}\mdseries}

% Page layout
\usepackage{geometry}
\geometry{a4paper,left=.8in,right=.6in,top=.75in,bottom=1in,heightrounded}
\usepackage{fancyhdr}
\fancyhf{}
\fancyhead[LE,RO]{\thepage}
\fancyhead[LO]{\nouppercase{\rightmark}}
\fancyhead[RE]{\nouppercase{\leftmark}}
\renewcommand{\headrulewidth}{0.5pt}
\renewcommand{\footrulewidth}{0pt}

% Chapter formatting
\usepackage{titlesec}
\titleformat{\chapter}[display]
{\normalfont\sffamily\Huge\bfseries\color{titleblue}}{\chaptertitlename\ \thechapter}{20pt}{\Huge}
\titleformat{\section}
{\normalfont\sffamily\LARGE\bfseries\color{titleblue!100!gray}}{\thesection}{1em}{}
\titleformat{\subsection}
{\normalfont\sffamily\Large\bfseries\color{titleblue!75!gray}}{\thesubsection}{1em}{}
\titleformat{\subsubsection}
{\normalfont\sffamily\large\bfseries\color{titleblue!75!gray}}{\thesubsection}{1em}{}

% Table of contents formatting
\usepackage{tocloft}
\renewcommand{\cftchapfont}{\sffamily\color{titleblue}\bfseries}
\renewcommand{\cftsecfont}{\sffamily\color{chaptergray}}
\renewcommand{\cftsubsecfont}{\sffamily\color{sectiongray}}
\renewcommand{\cftchapleader}{\cftdotfill{\cftdotsep}}

% TikZ
\usepackage{tikz}
\usepackage{tikz-cd}
\usetikzlibrary{math} % for calculations
\usetikzlibrary{matrix, positioning, arrows.meta, calc, shapes.geometric, shapes.multipart, chains}
\usetikzlibrary{decorations.pathreplacing,calligraphy}

\usepackage{pgfplots}
\usepgfplotslibrary{fillbetween}
\pgfplotsset{compat=1.15}

% Pseudo-code
\usepackage[linesnumbered,ruled]{algorithm2e}
\usepackage{algpseudocode}
\usepackage{setspace}
\SetKwComment{Comment}{/* }{ */}
\SetKwProg{Fn}{Function}{:}{end}
\SetKw{End}{end}
\SetKw{DownTo}{downto}

% Define a new environment for algorithms without line numbers
\newenvironment{algorithm2}[1][]{
	% Save the current state of the algorithm counter
	\newcounter{tempCounter}
	\setcounter{tempCounter}{\value{algocf}}
	% Redefine the algorithm numbering (remove prefix)
	\renewcommand{\thealgocf}{}
	\begin{algorithm}
	}{
	\end{algorithm}
	% Restore the algorithm counter state
	\setcounter{algocf}{\value{tempCounter}}
}

%Tcolorbox
\usepackage[most]{tcolorbox}
\tcbset{colback=white, arc=5pt}

% Listings
\usepackage{listings}
\lstdefinestyle{coq}{
	language=C,
	backgroundcolor=\color{white},
	basicstyle=\ttfamily\color{black},
	commentstyle=\color{green!70!black},
	keywordstyle={\bfseries\color{purple}},
	keywordstyle=[2]{\bfseries\color{red}},
	keywordstyle=[3]{\bfseries\color{type}},
	keywordstyle=[4]{\bfseries\color{JungleGreen}},
	keywordstyle=[5]{\bfseries\color{Magenta}},
	keywordstyle=[6]{\bfseries\color{RoyalBlue}},
	keywordstyle=[7]{\bfseries\color{Turquoise}},
	otherkeywords={bool, inline, size\_t, FILE},
	morekeywords=[2]{;},
	morekeywords=[3]{i8, i32, i64, u8, u32, u64, field, word},
	morekeywords=[4]{
		rdtsc, measure\_cycle, measure\_time, stringToWord
	},
	morekeywords=[5]{
		main
	},
	morekeywords=[6]{false, true, strlen, sscanf, printf},
	morekeywords=[7]{
		ONE, SIZE, IS\_32\_BIT\_ENV, PRIME, PRIME\_INVERSE
	},
	numberstyle=\tiny\color{gray},
	stringstyle=\color{purple},
	showstringspaces=false,
	breaklines=true,
	frame=single,
	framesep=3pt,
	%frameround=tttt,
	framexleftmargin=3pt,
	numbers=left,
	numberstyle=\small\color{gray},
	xleftmargin=15pt, % Increase the left margin
	xrightmargin=5pt,
	tabsize=4,
	belowskip=0pt,
	aboveskip=4pt
}
\lstdefinestyle{zsh}{
	language=bash,                  % Set the language to bash (closest to Zsh)
	backgroundcolor=\color{rustBack},
	commentstyle=\color{commentColor}\ttfamily,
	keywordstyle=\color{rustBuild}\bfseries,
	stringstyle=\color{stringColor!70!white}\ttfamily,
	keywordstyle=[2]{\bfseries\color{rustBuild}},
	keywordstyle=[3]{\bfseries\color{CLIGreen}},
	showspaces=false,               % Don't show spaces as underscores
	showstringspaces=false,         % Don't highlight spaces in strings
	breaklines=true,                % Automatic line breaking
	frame=none,                     % No frame around the code
	basicstyle=\ttfamily\color{white}, % White basic text color for contrast
	extendedchars=true,             % Allow extended characters
	captionpos=b,                   % Caption-position at bottom
	escapeinside={\%*}{*)},         % Allow LaTeX inside your code
	morekeywords={echo,ls,cd,pwd,exit,clear,man,unalias,zsh,source,
					cargo}, % Add more keywords
	upquote=true,                   % Ensure straight quotes are used
	literate=
		{\$}{{\textcolor{vsCodeRed}{$\boldsymbol{\$}$}}}1
		{>}{{\textcolor{CLIGreen}{$\boldsymbol{>}$}}}1
		{<}{{\textcolor{CLIGreen}{$\boldsymbol{<}$}}}1
		{@}{{\textcolor{vsCodeRed}{@}}}1
		{~}{{\textcolor{vsCodeRed}{$\boldsymbol{\sim}$}}}1
		{:}{{\textcolor{rustBuild}{:}}}1,
%		{:}{{\textcolor{red}{:}}}1
%		{~}{{\textcolor{red}{\textasciitilde}}}1 % Color certain characters
%		{├}{{\textbrokenbar}}1  % Replace ├ with \textbrokenbar
%		{─}{{\textendash}}1     % Replace ─ with \textendash
%		{└}{{`}}1,              % Replace └ with `
	morekeywords=[2]{cryptol, load, prove,},
	morekeywords=[3]{Main}
}

% Usage: \lstinputlisting[style=zsh]{sourcefile.sh}
% or \begin{lstlisting}[style=zsh] ... \end{lstlisting}


% Table
\usepackage{booktabs}
\usepackage{tabularx}
\usepackage{multicol}
\usepackage{multirow}
\usepackage{array}
\usepackage{longtable}
\newcolumntype{C}[1]{>{\centering\arraybackslash}p{#1}}

% Hyperlinks
\usepackage{hyperref}
\hypersetup{
	colorlinks=true,
	linkcolor=titleblue,
	filecolor=black,      
	urlcolor=titleblue,
}

%Ceiling and Floor Function
\usepackage{mathtools}
\DeclarePairedDelimiter{\ceil}{\lceil}{\rceil}
\DeclarePairedDelimiter{\floor}{\lfloor}{\rfloor}


\newcommand{\mathcolorbox}[2]{\colorbox{#1}{$\displaystyle #2$}}

\newcommand{\of}[1]{\left(#1\right)}

\newcommand{\N}{\mathbb{N}}
\newcommand{\Z}{\mathbb{Z}}
\newcommand{\Q}{\mathbb{Q}}
\newcommand{\R}{\mathbb{R}}
\newcommand{\C}{\mathbb{C}}
\newcommand{\F}{\mathbb{F}}
\newcommand{\zero}{\textcolor{red}{\texttt{0}}}
\newcommand{\one}{\textcolor{red}{\texttt{1}}}
\newcommand{\binaryfield}{\set{\zero,\one}}

\newcommand{\ie}{\textnormal{i.e.}}
\newcommand{\sol}{\textcolor{magenta}{\bf Solution.\ }}

\newcommand{\yes}{\textcolor{blue}{O}}
\newcommand{\no}{\textcolor{red}{X}}

\newcommand{\adjacent}{\parallel}

\newcommand{\src}{\mathsf{src}}
\newcommand{\dsc}{\mathsf{dsc}}
\newcommand{\state}{\mathsf{state}}
\newcommand{\data}{\mathsf{data}}

\newcommand{\len}{\mathsf{len}}
\newcommand{\bit}{\mathsf{bit}}
\newcommand{\byte}{\mathsf{byte}}
\newcommand{\word}{\mathsf{word}}

\begin{document}	
	% Title page
	\begin{titlepage}
		\begin{center}
			{\Huge\textsf{\textbf{Coq: A Comprehensive Guide}}\par}
			{\Large\textsf{\textbf{- Mastering the Art of Coq Programming -}}\par}
			\vspace{0.5in}
			{\Large {Ji Yong-Hyeon}\par}
			\vspace{1in}
%			\includegraphics[scale=.6]{images/AES_Encryption_Round.png}\par
			\vspace{1in}
			{\large\bf \textsf{Department of Information Security, Cryptology, and Mathematics}\par}
			{\textsf{College of Science and Technology}\par}
			{\textsf{Kookmin University}\par}
			\vspace{.25in}
			{\large \textsf{\today}\par}
		\end{center}
	\end{titlepage}
	
	\frontmatter
	\section*{Data}
%	\section*{Abstract}

This book is about...

% Your abstract here

	
	\newpage
	\tableofcontents
	
	\newpage
	\mainmatter
	\chapter{Coq Tutorial} %
%	\input{chapters/ecc_ch1_mul_squ.tex} %
%	\input{chapters/ecc_ch1_red.tex} %
%	\input{chapters/chapter1.tex}
%	\input{chapters/chapter2.tex}
	% Include more chapters as needed
	
	\appendix
	\begin{thebibliography}{9}
	\bibitem{textbook}
	test
\end{thebibliography} 
%	\chapter{Additional Data A}

\section{Existence of an Additional Root in Cubic Functions via the Intermediate Value Theorem}

\textbf{Theorem:} Let \( f(x) = ax^3 + bx^2 + cx + d \) be a cubic function, where \( a, b, c, d \in \mathbb{R} \) and \( a \neq 0 \). If \( x_1 \) and \( x_2 \) are two distinct roots of \( f(x) \), there exists at least one other root \( x_3 \) of \( f(x) \).

\textbf{Proof:}

\begin{enumerate}
	\item \textit{Cubic Function}: A cubic function is defined as \( f(x) = ax^3 + bx^2 + cx + d \), which is a polynomial of degree 3, and thus continuous over \(\mathbb{R}\).
	
	\item \textit{Known Roots}: Assume \( x_1 \) and \( x_2 \) are two distinct roots of \( f(x) \), i.e., \( f(x_1) = f(x_2) = 0 \).
	
	\item \textit{Intermediate Value Theorem (IVT)}: The IVT states that for any continuous function \( g \) on an interval \([a, b]\), if \( g(a) \) and \( g(b) \) have opposite signs, there is at least one \( c \) in \((a, b)\) such that \( g(c) = 0 \).
	
	\item \textit{Application to Cubic Function}: By the nature of cubic functions, they must change direction at least once between two roots. This implies the function will either attain a local maximum or minimum between \( x_1 \) and \( x_2 \).
	
	\item \textit{Existence of Third Root}: If the local extremum is above or below the x-axis, the function must cross the x-axis to change direction, implying the existence of another root \( x_3 \) in the interval \((x_1, x_2)\).
	
	\item \textit{Conclusion}: Therefore, by drawing a straight line through \( (x_1, 0) \) and \( (x_2, 0) \), this line will intersect the graph of \( f(x) \) at least at one other point, indicating the existence of another root \( x_3 \).
\end{enumerate}

\begin{tikzpicture}
	\begin{axis}[
		xlabel={$x$},
		ylabel={$f(x)$},
		axis lines=middle,
		xmin=-3, xmax=3,
		ymin=-3, ymax=3,
		grid=both,
		xtick={-2, 0, 2},
		ytick={-2, 0, 2},
		xticklabels={$x_1$, $0$, $x_2$},
		yticklabels={,,}
		]
		
		% Cubic function
		\addplot[smooth, thick, red] expression[domain=-2:2, samples=100]{x^3 - 3*x};
		
		% Dots at roots
		\addplot[only marks, mark=*, mark options={fill=blue}] coordinates {(-2,0) (2,0)};
		
		% Optional: Add a point for the third root
		\addplot[only marks, mark=*, mark options={fill=green}] coordinates {(0,0)};
		
		% Labels for the roots
		\node[below] at (axis cs:-2,0) {$x_1$};
		\node[below] at (axis cs: 2,0) {$x_2$};
		\node[above] at (axis cs: 0,0) {$x_3$};
		
	\end{axis}
\end{tikzpicture}

\begin{tikzpicture}
	\begin{axis}[
		xlabel={$x$},
		ylabel={$f(x)$},
		axis lines=middle,
		xmin=-3, xmax=3,
		ymin=-5, ymax=5,
		grid=both
		]
		
		% Define a cubic function
		\addplot[smooth, thick, red, domain=-3:3, samples=100] {x^3 - 3*x};
		
		% Mark the known roots
		\addplot[only marks, mark=*, mark options={fill=blue}] coordinates {(-sqrt(3),0) (sqrt(3),0)};
		
		% Draw a straight line through the two known roots
		\addplot[thick, blue, domain=-3:3] {0};
		
%		% Label for roots and potential third root
%		\node[below] at (axis cs:-sqrt(3),0) {$x_1$};
%		\node[below] at (axis cs: sqrt(3),0) {$x_2$};
%		\node[above] at (axis cs: 0,0) {Possible $x_3$};
		
	\end{axis}
\end{tikzpicture}

\begin{tikzpicture}[scale=0.5, >=Stealth]
	
	% Define the range for drawing
	\foreach \x in {-5,...,5} {
		\foreach \y in {-5,...,5} {
			% Place a dot at each point
			\node[draw,circle,inner sep=1pt,fill] at (\x,\y) {};
		}
	}
	
	% Draw vectors
	\draw[thick,->] (-4,-3) -- (4,-1) node[midway, below] {$\mathbf{v}_1$};
	\draw[thick,->] (-4,-3) -- (-3,1) node[midway, above] {$\mathbf{v}_2$};
	
	% Optional: Draw coordinate axes
	\draw[thin,->] (-6,0) -- (6,0) node[right] {$x$};
	\draw[thin,->] (0,-6) -- (0,6) node[above] {$y$};
	
\end{tikzpicture}

% Appendix A content

%	\input{appendices/appendixB.tex}
	
	\backmatter
	% Bibliography, index, etc.
	
\end{document}